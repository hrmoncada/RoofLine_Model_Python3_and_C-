% !TEX root = ../thesis-sample.tex
\appendix
\doublespacing
%\chapter{Appendix}

\section{Front-end and Back-end}
In software engineering, the terms \verb+front-end+ and \verb+back-end+ refer to the separation
of concerns between the presentation layer (\verb+front-end+), and the data access layer (\verb+back-end+) of a piece of software, 
or the physical infrastructure or hardware. Therefore in a HPC, 
\begin{itemize}
\item The login servers are called \verb+front-ends+ because you do not run your calculations there.
\item Rather run your calculations on \verb+back-end+ compute servers.
\item The \verb+front-end+ server provides access to compute servers via the \verb+batch+ system, using the \verb+qsub+ command.
\end{itemize}

For example, interactive HPC entails a lot of new challenges that have to be solved one of them addressing
the fast and efficient data transfer between a simulation \verb+back end+ and visualisation \verb+front end+,
as several gigabytes of data per second are nothing unusual for a simulation running on some (hundred) thousand cores.
\begin{itemize}
\item The login on Grizzly or Badger are called \verb+front-ends+ because you do not run your calculations there.
\item Request a node allocation using \verb+salloc+ will put you on \verb+back-end+ compute cluster, here you can run your calculations on.
\end{itemize}

\section{Hard Link or Symbolic Link}
The \verb+ln+ command is a \verb+Linux/Unix+ command used to create file links to an existing file. 
\subsection{Link types}
\begin{itemize}
\item There are two types of links
\begin{enumerate}
\item  \textbf{hard links:} Refer to the specific location of physical data.
A hard link allows multiple filenames to be associated with the same file since a hard link points to the 
inode of a given file, the data of which is stored on disk.
\item  \textbf{symbolic links:} Refer to a symbolic path indicating the abstract location of another file.
A symbolic links are special files that refer to other files by name.
\end{enumerate}
\item The \verb+ln+ command by default creates hard links, and when called with the command line parameter \verb+ln -s+
creates symbolic links.
\item Most operating systems prevent hard links to directories from being created since such a capability could disrupt
the structure of a file system and interfere with the operation of other utilities. 
\item The \verb+ln+ command can however be used to create symbolic links to  non-existent files. 
\end{itemize}

\subsection{Invoking Links} 
There are two ways to invoking the links
\begin{enumerate}
\item Single file invocation
\begin{lstlisting}[language=bash,numbers=none] 
ln [-fs] [-L|-P] source_file target_file
\end{lstlisting}
\item Multiple file invocation
\begin{lstlisting}[language=bash,numbers=none] 
ln [-fs] [-L|-P] source_file_1 source_file_2 ...source_file_n  target_dir
\end{lstlisting}
\end{enumerate}
The specification also specifies the command line options that must be supported:
\begin{itemize}
\item \verb+-f+ Force existing destination pathnames to be removed to allow the link.
\item \verb+-L+ For each \verb+source_file+ operand that names a file that is a symbolic link,
create a hard link to the file referenced by the symbolic link.
\item \verb+-P+ For each \verb+source_file+ operand that names a file that is a symbolic link, create a (hard)
link to the symbolic link itself.
\item \verb+-s+ Create symbolic links instead of hard links. If the \verb+-s+ option is specified, the \verb+-L+ and \verb+-P+
options are silently ignored.
\item If more than one of the mutually-exclusive options \verb+-L+ and \verb+-P+ is specified the last option specified determines the
behavior of the utility.
\item If the \verb+-s+ option is not specified and neither a \verb+-L+ nor a \verb+-P+ option is specified, the implementation defines
which of the \verb+-L+ and \verb+-P+ options will be used as the default.
\end{itemize}

\subsection{Examples}
\begin{enumerate}
 \item Example 1
\begin{lstlisting}[language=bash,numbers=none] 
$ ln -s source_file target_file
\end{lstlisting}
\begin{lstlisting}[language=bash,numbers=none] 
$ ls -l source_file target_file
-rw-r--r--  1 veryv  wheel  0 Mar  7 22:01 source_file
lrwxr-xr-x  1 veryv  wheel  5 Mar  7 22:01 target_file -> source_file
\end{lstlisting} 
\item Example 2 - Create a symbolic link for \verb+/home/Desktop/Links/Example/example.cpp+ as \verb+/home/Test/example.cpp+,
copy paste the following command
\begin{lstlisting}[language=bash,numbers=none] 
$ ln -s   /home/Desktop/Links/Example/example.cpp    /home/Test/example.cpp
\end{lstlisting}
\begin{lstlisting}[language=bash,numbers=none] 
$ ll
lrwxrwxrwx 1 vivek  vivek    16 2007-09-25 22:53 example.cpp -> /home/Desktop/Links/Example/example.cpp
\end{lstlisting}
\end{enumerate}

\section{Securely Copy (SCP) Files}
\verb+SCP+ allows files to be copied to, from, or between different hosts (between a local host and a remote host or between two remote hosts.).
It uses \verb+ssh+ for data transfer and provides the same authentication and same level of security as \verb+ssh+.
\begin{enumerate}
 \item Copy the file \verb+foobar.txt+ from a remote host to the local host
\begin{lstlisting}[language=bash,numbers=none] 
$ scp your_username@remotehost.edu:foobar.txt /some/local/directory 
\end{lstlisting}
\item How to \verb+scp+ a file to LANL-IC \verb+turquoise/+
\begin{lstlisting}[language=bash,numbers=none] 
$ scp filename username@wtrw.lanl.gov:username@gr-fe.lanl.gov:/remote/path/to/file
\end{lstlisting}
\item Also shorter probably works
\begin{lstlisting}[language=bash,numbers=none] 
scp filename wtrw:gr-fe:/remote/path/to/file
\end{lstlisting}
\begin{lstlisting}[language=bash,numbers=none] 
scp *.tar.gz hrmoncada@wtrw.lanl.gov:hrmoncada@gr-fe.lanl.gov:/usr/projects/climate/hrmoncada/repos/Performance
\end{lstlisting}
\end{enumerate}

\subsection{SCP examples} 
%https://haydenjames.io/linux-securely-copy-files-using-scp/
\begin{itemize}
\item Copy file from a remote host to local host SCP example:
\begin{lstlisting}[language=bash,numbers=none] 
$ scp username@from_host:file.txt /local/directory/
\end{lstlisting}
\item Copy file from local host to a remote host SCP example:
\begin{lstlisting}[language=bash,numbers=none] 
$ scp file.txt username@to_host:/remote/directory/
\end{lstlisting} 
\item Copy directory from a remote host to local host SCP example:
\begin{lstlisting}[language=bash,numbers=none] 
$ scp -r username@from_host:/remote/directory/  /local/directory/
\end{lstlisting}
\item Copy directory from local host to a remote hos SCP example:
\begin{lstlisting}[language=bash,numbers=none] 
$ scp -r /local/directory/ username@to_host:/remote/directory/
\end{lstlisting} 
\item Copy file from remote host to remote host SCP example:
\begin{lstlisting}[language=bash,numbers=none] 
$ scp username@from_host:/remote/directory/file.txt username@to_host:/remote/directory/
\end{lstlisting}
\end{itemize}
\subsection{Notes}
\begin{itemize}
\item SCP example:  
\begin{lstlisting}[language=bash,numbers=none] 
scp -r  root@123.123.123.123:/var/www/html/ /home/hydn/backups/test/ Also see: Backup solutions.
\end{lstlisting}
\item Host can be IP or domain name. Once you click return, you will be prompted for SSH password.
\item Although this page covers SCP Linux, the instructions will also work for Mac using ``Terminal''.
You can also use WinSCP to accomplish this on a Windows PC/server.
\item When copying a source file to a target file which already exists, scp will replace the contents of the target file. So be careful.
\end{itemize}
\subsection{SCP options}
\begin{itemize}
\item \verb+-r+ Recursively copy entire directories. Note that this follows symbolic links encountered in the tree traversal.
\item \verb+-C+ Compression enable. Passes the \verb+-C+ flag to ssh to enable compression.
\item \verb+-l+ limit - Limits the used bandwidth, specified in \verb+Kbit/s+.
\item \verb+-+o \verb+ssh_option+ - Can be used to pass options to ssh in the format used in \verb+ssh_config+.
\item \verb+-P+ port - Specifies the port to connect to on the remote host. Note that this option is written with a capital \verb+P+.
\item \verb+-p+ Preserves modification times, access times, and modes from the original file.
\item \verb+-q+ Quiet mode: disables the progress meter as well as warning and diagnostic messages from \verb+ssh+.
\item \verb+-v+ Verbose mode. Print debugging messages about progress. 
This is helpful in debugging connection, authentication, and configuration problems.
\end{itemize}

\subsection{Desktop to LANL-IC Turquoise}
\begin{itemize}
 \item How to \verb+scp+ a file to \verb+LANL- IC Turquoise+:
\begin{lstlisting}[language=bash,numbers=none] 
scp filename hrmoncada@wtrw.lanl.gov:hrmoncada@gr-fe.lanl.gov:/remote/path/to/file
\end{lstlisting}
example, transfer the  file \verb+testcode.tar.gz+
\begin{lstlisting}[language=bash,numbers=none,basicstyle=\scriptsize] 
$ scp testcode.tar.gz hrmoncada@wtrw.lanl.gov:hrmoncada@gr-fe.lanl.gov:/usr/projects/climate/hrmoncada/repos/debug_examples
hrmoncada@wtrw.lanl.gov's password: 
testcode.tar.gz                                                               100% 1124     1.1KB/s   00:00
\end{lstlisting}
example, transfer the  file \verb+example_1.tar.gz+
\begin{lstlisting}[language=bash,numbers=none,basicstyle=\scriptsize] 
$ scp example_1.tar.gz hrmoncada@wtrw.lanl.gov:hrmoncada@gr-fe.lanl.gov:/usr/projects/climate/hrmoncada/repos/debug_examples
hrmoncada@wtrw.lanl.gov's password: 
testcode.tar.gz                                                               100% 1124     1.1KB/s   00:00
\end{lstlisting}
\item also shorter probably works:
\begin{lstlisting}[language=bash,numbers=none] 
scp filename wtrw:gr-fe:/remote/path/to/file
\end{lstlisting}
example, transfer the  file \verb+example.tar+
\begin{lstlisting}[language=bash,numbers=none,basicstyle=\scriptsize] 
$ scp examples.tar wtrw:gr-fe:/usr/projects/climate/hrmoncada/repos/debug_examples
The authenticity of host 'wtrw (204.121.65.1)' can't be established.
RSA key fingerprint is SHA256:FssxNwJeTvPbeFHU6z/TaLvXGPfgE/RCp2LmBQzdlBs.
Are you sure you want to continue connecting (yes/no)? yes
Warning: Permanently added 'wtrw' (RSA) to the list of known hosts.
hrmoncada@wtrw's password: 
examples.tar                                                                 100%   20KB  20.0KB/s   00:00  
\end{lstlisting}

\end{itemize}
%\chapter{Another Appendix}
%\lipsum[24]
